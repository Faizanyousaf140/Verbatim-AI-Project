\documentclass[12pt]{article}
\usepackage[utf8]{inputenc}
\usepackage[T1]{fontenc}
\usepackage[a4paper, margin=1in]{geometry}
\usepackage{amsmath}
\usepackage{graphicx}
\usepackage{hyperref}
\usepackage{enumitem}
\usepackage{xcolor}
\usepackage{listings}
\usepackage{caption}
\usepackage{subcaption}
\usepackage{float}
\usepackage{multirow}
\usepackage{array}
\usepackage{booktabs}
\usepackage{natbib}
\usepackage{notoserif} % Reliable serif font for English and other languages
\usepackage{titlesec}
\usepackage{tocloft}
\usepackage{fancyhdr}

% Preamble configuration for enhanced PDF
\hypersetup{
    colorlinks=true,
    linkcolor=blue,
    filecolor=magenta,
    urlcolor=cyan,
    linktoc=all % Makes entire ToC entries clickable
}
\lstset{
    basicstyle=\ttfamily\small,
    breaklines=true,
    frame=single,
    captionpos=b,
}
% Customize section headings
\titleformat{\section}{\Large\bfseries}{\thesection}{1em}{}
\titleformat{\subsection}{\large\bfseries}{\thesubsection}{1em}{}
% Header and footer
\pagestyle{fancy}
\fancyhf{}
\fancyhead[L]{VerbatimAI SRS}
\fancyhead[R]{Page \thepage}
\fancyfoot[C]{\today, 03:16 PM PKT}

\title{Software Requirements Specification (SRS) for VerbatimAI}
\author{Ahsan \& Faizan}
\date{August 06, 2025}

\begin{document}

\maketitle

\begin{abstract}
This Software Requirements Specification (SRS) document provides an exhaustive blueprint for VerbatimAI, an enterprise-grade meeting intelligence platform developed using Python Streamlit. VerbatimAI automates meeting transcription, analysis, and insight generation, addressing inefficiencies in manual note-taking for remote teams, HR professionals, legal departments, and educators. This enhanced document details the system's purpose, scope, comprehensive functional and non-functional requirements, intricate architecture, detailed use cases, UI navigation, implementation challenges, and supporting diagrams, serving as a definitive guide for development, testing, and deployment.
\end{abstract}

% Navigable Table of Contents
\tableofcontents
\newpage

\section{Introduction}
\subsection{Purpose}
VerbatimAI is a Streamlit-based platform engineered to revolutionize meeting documentation by automating transcription, extracting actionable insights (decisions, action items, questions), detecting sentiments, and enabling semantic search with 95%+ transcription accuracy. The system aims to enhance productivity, minimize manual effort, ensure data integrity, and provide real-time analytics, catering to a diverse user base including remote teams, HR, legal, and educational sectors. It replaces traditional note-taking with a robust, scalable solution, supporting enterprise-grade operations.

\subsection{Scope}
VerbatimAI encompasses an extensive set of functionalities, including multi-format audio/video upload (MP3, WAV, MP4, AVI, MOV, M4A up to 1GB) and real-time recording with live preview, transcription with speaker diarization, AI-powered key point extraction using spaCy/BERT, sentiment analysis with YAMNet/CREMA-D models, interactive analytics dashboards with Plotly charts, semantic search using sentence-transformers/FAISS, and professional exports (PDF, DOCX, CSV, JSON, email summaries). The system is optimized for meeting types such as client interviews, technical discussions, standups, and performance reviews. Excluded features include multilingual support, mobile apps, offline mode, and third-party integrations (e.g., Zoom, Slack), which are deferred to future phases pending resource allocation and market demand.

\subsection{Definitions, Acronyms, and Abbreviations}
\begin{itemize}
    \item \textbf{ARS}: Airplane Reservation System (not applicable here).
    \item \textbf{ASR}: Automatic Speech Recognition.
    \item \textbf{NLP}: Natural Language Processing.
    \item \textbf{UI}: User Interface.
    \item \textbf{API}: Application Programming Interface.
    \item \textbf{FAISS}: Facebook AI Similarity Search.
    \item \textbf{DFD}: Data Flow Diagram.
    \item \textbf{UML}: Unified Modeling Language.
    \item \textbf{WCAG}: Web Content Accessibility Guidelines.
\end{itemize}

\subsection{References}
\begin{itemize}
    \item AssemblyAI Documentation: \url{https://www.assemblyai.com/docs/}
    \item Streamlit Documentation: \url{https://docs.streamlit.io/}
    \item Sentence Transformers: \url{https://www.sbert.net/}
    \item IEEE Std 830-1998, IEEE Recommended Practice for Software Requirements Specifications.
    \item WCAG 2.1 Guidelines: \url{https://www.w3.org/TR/WCAG21/}
\end{itemize}

\subsection{Overview}
This SRS offers a detailed roadmap for VerbatimAI, encompassing its purpose, user interactions, technical requirements, UI navigation, a comprehensive set of diagrams, and operational constraints. It serves as a contractual agreement between stakeholders (developers, testers, and management) and end-users, ensuring alignment on system objectives, limitations, and implementation strategies. The document is structured to facilitate iterative development, with sections linked for easy navigation via the Table of Contents.

\section{Overall Description}
\subsection{Product Perspective}
VerbatimAI is a standalone Streamlit application that supersedes manual meeting documentation tools and basic recording systems. It integrates with AssemblyAI for high-accuracy transcription, sentence-transformers/FAISS for semantic search, Celery/Redis for asynchronous processing, and python-docx/reportlab for professional exports. The system is designed as a microservices architecture, deployable on Docker/Kubernetes, offering a scalable alternative to legacy solutions with enhanced data security and user experience.

\subsection{Product Functions}
\begin{itemize}
    \item Multi-format audio/video upload and real-time recording with live preview and synchronized output.
    \item Real-time and post-meeting transcription with speaker diarization and confidence scoring.
    \item AI-powered extraction of decisions, action items, questions, entities (names, dates), and topic clusters.
    \item Per-speaker sentiment analysis with emotion detection (calm, angry, confident) and trend visualization.
    \item Interactive analytics dashboard with multi-tab views (Speaker, Sentiment, Semantic, Engagement, AI, Reports) and exportable Plotly charts.
    \item Semantic search with vector-based similarity, multi-dimensional filtering (speaker, sentiment, time, topics), and cross-meeting analysis.
    \item Professional exports in PDF, DOCX, CSV, JSON with batch processing, and automated email summaries with customizable templates.
    \item Background processing with Celery/Redis for scalable task management and real-time progress tracking.
    \item Role-based access control (admin, viewer, contributor) with user management and 2FA for admins.
\end{itemize}

\subsection{User Classes and Characteristics}
\begin{itemize}
    \item \textbf{Remote Teams/Project Managers}: Require efficient meeting documentation, action tracking, and collaboration tools (tech-savvy, 25-50 years, frequent users).
    \item \textbf{HR Professionals}: Need sentiment analysis, engagement metrics, and compliance reports (moderate tech skills, 30-55 years, occasional users).
    \item \textbf{Legal/Compliance}: Demand accurate transcripts, entity extraction, and secure exports (high attention to detail, 35-60 years, infrequent but critical users).
    \item \textbf{Educators}: Seek lecture transcription, student participation analysis, and study material generation (varied tech skills, 30-65 years, seasonal users).
\end{itemize}

\subsection{Operating Environment}
VerbatimAI operates on modern browsers (Chrome v90+, Firefox v80+, Edge v88+) with a Streamlit backend on Python 3.10+, requiring 4GB RAM (8GB recommended for video processing), 2GB storage (scalable to 10GB with large datasets), and a stable internet connection (minimum 10 Mbps). Supported operating systems include Windows 10/11, macOS 10.15+, and Ubuntu 18.04+, with hardware compatibility for PCs (Intel i3+ or equivalent), smartphones (Android 10+, iOS 13+ with 2GB RAM), and webcams/microphones (640x480 resolution, 16-bit audio, 44.1 kHz sample rate).

\subsection{Design and Implementation Constraints}
\begin{itemize}
    \item Must comply with WCAG 2.1 for accessibility (e.g., screen reader support, high-contrast themes).
    \item Relies on Streamlit for UI, limiting support for complex animations or native mobile apps.
    \item Supports files up to 1GB with chunked processing to manage memory usage.
    \item Requires AssemblyAI API for transcription and Redis server for queue management.
    \item Enforces data encryption (AES-256) and secure API communication (HTTPS/TLS 1.3).
\end{itemize}

\subsection{Assumptions and Dependencies}
\begin{itemize}
    \item Assumes stable internet connectivity and uninterrupted camera/microphone access.
    \item Depends on AssemblyAI API availability (99.9% uptime) and Redis server performance.
    \item Assumes users possess basic digital literacy and browser compatibility (JavaScript enabled).
    \item Relies on third-party libraries (e.g., OpenCV, PyAudio) for real-time recording functionality.
\end{itemize}

\subsection{User Interface Navigation}
The VerbatimAI UI is accessible via a Streamlit sidebar with the following navigable sections, each featuring sub-options and real-time feedback:
\begin{itemize}
    \item \textbf{Home} (\emoji{🏠}): Overview page with quick start guide, system status, and welcome message.
    \item \textbf{Record Meeting} (\emoji{🎙️🎥}): Upload files (MP3, WAV, MP4, AVI, MOV, M4A up to 1GB), start/stop real-time recording, pause/resume controls, and live preview window.
    \item \textbf{Analytics Dashboard} (\emoji{📊}): Multi-tab interface with Speaker Analysis (time distribution, engagement scores), Sentiment Trends (emotion heatmaps), Semantic Insights (topic clusters), Engagement Metrics (participation balance), AI Highlights (key points), and Reports (exportable summaries).
    \item \textbf{Semantic Search} (\emoji{🔍}): Search bar with autocomplete, filter options (speaker, sentiment, time, topics), and ranked result display with relevance scores.
    \item \textbf{Meeting Library} (\emoji{📚}): Browse saved meetings, access transcripts, view analytics, and delete outdated records.
    \item \textbf{Export \& Share} (\emoji{📄}): Generate reports (PDF, DOCX, CSV, JSON), configure email summaries (recipient, template), and initiate batch exports.
    \item \textbf{Settings} (\emoji{⚙️}): Manage user profile, toggle dark/light themes, input API keys, adjust recording settings (sample rate, resolution), and enable 2FA.
\end{itemize}
Each section includes interactive elements (e.g., \texttt{st.button()}, \texttt{st.slider()}) and error handling with user-friendly messages.

\section{Specific Requirements}
\subsection{Functional Requirements}
\begin{enumerate}
    \item \textbf{File Upload and Recording}
    \begin{itemize}
        \item The system shall allow users to upload MP3, WAV, MP4, AVI, MOV, M4A files up to 1GB using \texttt{st.file_uploader()} with progress bars.
        \item The system shall support real-time audio/video recording with live preview using OpenCV (video) and PyAudio (audio), controllable via \texttt{st.button()} for start/stop and pause/resume.
        \item The system shall synchronize audio and video streams with a configurable maximum duration (default 1 hour, adjustable via \texttt{st.slider()}), saving outputs as WAV/MP4 files.
        \item The system shall validate file formats and sizes, displaying errors (e.g., "Unsupported format") if invalid.
    \end{itemize}
    \item \textbf{Transcription}
    \begin{itemize}
        \item The system shall provide real-time transcription with 95%+ accuracy using AssemblyAI API, updating every 5 seconds via WebSocket.
        \item The system shall perform post-meeting transcription with batch processing for uploaded files, completing within 1-3 minutes for 1GB files.
        \item The system shall implement speaker diarization with confidence scoring (0-100%) displayed in transcripts, using color-coded labels (e.g., Speaker A: 92%).
        \item The system shall optimize transcription for meeting types (interviews, standups, reviews) with predefined templates selectable via \texttt{st.selectbox()}.
        \item The system shall handle transcription failures (e.g., API downtime) with fallback to cached data and user notifications.
    \end{itemize}
    \item \textbf{Key Point Extraction}
    \begin{itemize}
        \item The system shall extract decisions, action items, and questions using spaCy/BERT, highlighting them in transcripts with \texttt{st.markdown()} and tooltips.
        \item The system shall categorize technical discussions and extract entities (names, dates, organizations) with 90% accuracy, storing results in a searchable database.
        \item The system shall support topic clustering and progression analysis, visualized as timelines in the Analytics Dashboard.
        \item The system shall allow manual correction of extracted points via an interactive editor in the Meeting Library.
    \end{itemize}
    \item \textbf{Sentiment Analysis}
    \begin{itemize}
        \item The system shall detect emotions (calm, angry, confident) per speaker using YAMNet/CREMA-D models, updating in real-time during recording.
        \item The system shall provide sentiment trends with confidence scores (0-100%) via interactive Plotly heatmaps in the Analytics Dashboard.
        \item The system shall generate per-speaker sentiment reports, exportable as CSV, with historical trend comparison.
        \item The system shall handle noisy audio inputs with pre-processing (e.g., noise reduction) to improve accuracy.
    \end{itemize}
    \item \textbf{Analytics Dashboard}
    \begin{itemize}
        \item The system shall display speaker metrics (total time, engagement score, word count) in a three-column layout using \texttt{st.columns()}.
        \item The system shall provide interactive Plotly charts (pie charts for speaking time, line charts for sentiment) in a multi-tab interface.
        \item The system shall generate heatmaps and timelines for sentiment evolution and topic progression, exportable in PNG/SVG.
        \item The system shall support custom date ranges and speaker filters via \texttt{st.date_input()} and \texttt{st.multiselect()}.
    \end{itemize}
    \item \textbf{Semantic Search}
    \begin{itemize}
        \item The system shall enable vector-based similarity search using sentence-transformers and FAISS, returning top-5 results with relevance scores (0-100%).
        \item The system shall support multi-dimensional filtering by speaker, sentiment, time (e.g., last 30 days), and topics via \texttt{st.form()}.
        \item The system shall perform similarity analysis across meetings, highlighting related content with \texttt{st.expander()}.
        \item The system shall index new transcripts automatically within 10 seconds of processing completion.
    \end{itemize}
    \item \textbf{Export and Automation}
    \begin{itemize}
        \item The system shall export transcripts as PDF (using reportlab), DOCX (using python-docx), CSV, JSON with customizable layouts (e.g., include charts).
        \item The system shall send automated email summaries via SMTP with templates editable via \texttt{st.text_area()}, scheduled daily/weekly.
        \item The system shall support batch export for multiple meetings, processing up to 10 files concurrently.
        \item The system shall encrypt exported files with user-defined passwords for security.
    \end{itemize}
    \item \textbf{Background Processing}
    \begin{itemize}
        \item The system shall use Celery/Redis for asynchronous task processing (transcription, analysis, export), supporting up to 50 concurrent tasks.
        \item The system shall provide real-time task tracking with WebSocket updates, displayed as a progress bar in the UI.
        \item The system shall implement task routing for optimal performance (e.g., high-priority transcription).
        \item The system shall log task failures with timestamps and retry mechanisms (up to 3 attempts).
    \end{itemize}
    \item \textbf{User Management}
    \begin{itemize}
        \item The system shall implement role-based access (admin, viewer, contributor) using \texttt{st.session_state}, with login via email/password.
        \item The system shall allow admins to manage user access, assign roles, and view activity logs via a dedicated tab.
        \item The system shall support 2FA for admins using time-based one-time passwords (TOTP).
        \item The system shall enforce password complexity (8+ characters, mix of types) and session timeouts (30 minutes inactivity).
    \end{itemize}
    \item \textbf{System Health Monitoring}
    \begin{itemize}
        \item The system shall perform daily health checks on API connections, Redis, and storage, alerting admins via email.
        \item The system shall display system status (uptime, load) on the Home page.
        \item The system shall generate monthly performance reports for admins.
    \end{itemize}
\end{enumerate}

\subsection{External Interface Requirements}
\subsubsection{User Interfaces}
A responsive Streamlit UI with a sidebar navigation bar, file uploaders, multi-tab analytics dashboard, real-time progress indicators, and accessibility features (e.g., keyboard navigation, high-contrast mode), optimized for desktop (1280x720+), tablet (800x600+), and mobile (360x640+) resolutions. Themes (dark/light) are toggleable via Settings.

\subsubsection{Hardware Interfaces}
Compatible with PCs (4GB RAM, Intel i3+ or equivalent), smartphones (Android 10+, iOS 13+ with 2GB RAM), and webcams/microphones (640x480 resolution, 16-bit audio, 44.1 kHz sample rate).

\subsubsection{Software Interfaces}
Integrates with AssemblyAI for transcription (REST API), sentence-transformers/FAISS for search (local models), Celery/Redis for queues (TCP), and python-docx/reportlab for exports (file I/O). Requires Python 3.10+ with dependencies (numpy, pandas, plotly).

\subsubsection{Communication Interfaces}
Uses HTTPS (TLS 1.3) for secure data transfer between client and server, SMTP (port 587) for email notifications, and WebSocket for real-time updates. Supports JSON payload format for API interactions.

\subsection{Non-Functional Requirements}
\subsubsection{Performance Requirements}
Responds in 1-3 minutes for 1GB file transcription, handles 100 concurrent users with <5s latency for real-time updates, and processes 10 batch exports simultaneously within 5 minutes.

\subsubsection{Safety Requirements}
Includes daily backups with Celery/Redis (encrypted storage), restores within 10 minutes, and disaster recovery plans for API outages.

####Security Requirements}
Encrypts data at rest with AES-256, in transit with TLS 1.3, complies with WCAG 2.1 (Level AA), and uses role-based access with 2FA for admins. Conducts quarterly security audits.

\subsubsection{Software Quality Attributes}
\begin{itemize}
    \item \textbf{Usability}: Intuitive design with tooltips, context-sensitive help, and 90% task completion rate for new users within 10 minutes.
    \item \textbf{Availability}: 99.9% uptime with redundancy across two AWS regions.
    \item \textbf{Maintainability}: Modular code with 80% test coverage, supporting updates without downtime.
    \item \textbf{Scalability}: Supports horizontal scaling with Docker/Kubernetes, handling 1,000 users with <10% performance degradation.
    \item \textbf{Reliability}: Achieves 99% transcription accuracy across 1,000 test cases.
\end{itemize}

\section{System Architecture}
\subsection{Three-Tier Architecture}
\begin{itemize}
    \item \textbf{Presentation Layer}: Streamlit UI with file uploaders, analytics dashboard, and navigation, hosted on Nginx.
    \item \textbf{Application Layer}: FastAPI for API routes, Celery for background processing, and WebSocket for real-time updates.
    \item \textbf{Data Layer}: Redis for caching/queues, local file system for storage, and PostgreSQL for user data (optional).
\end{itemize}

\subsection{Data Flow}
\begin{enumerate}
    \item \textbf{Input}: Audio/video file upload or real-time recording via UI, validated for format/size.
    \item \textbf{Processing}: AssemblyAI transcription, NLP analysis with Celery queues, and sentiment detection.
    \item \textbf{Storage}: Meeting library with metadata in Redis/file system, indexed for search.
    \item \textbf{Output}: Analytics dashboard rendered in Streamlit, exports generated via background tasks.
\end{enumerate}

\section{Use Cases}
\subsection{Client Interviews}
\begin{itemize}
    \item \textbf{Actors}: HR, Interviewers.
    \item \textbf{Description}: Record interview, transcribe audio, extract key insights (e.g., candidate responses), and analyze sentiment for hiring decisions.
    \item \textbf{Preconditions}: User logged in, microphone/camera enabled, internet active.
    \item \textbf{Postconditions}: Transcript saved in Meeting Library, sentiment report emailed to HR.
    \item \textbf{UI Section}: Record Meeting, Analytics Dashboard, Export \& Share.
    \item \textbf{Steps}:
        \begin{enumerate}
            \item User selects "Record Meeting" and starts audio/video capture.
            \item System transcribes in real-time, displaying speaker labels.
            \item User stops recording, reviews transcript, and extracts insights.
            \item System generates sentiment analysis and exports report.
        \end{enumerate}
    \item \textbf{Exceptions}: Microphone failure triggers error message and fallback to file upload.
\end{itemize}

\subsection{Technical Discussions}
\begin{itemize}
    \item \textbf{Actors}: Engineers, Project Managers.
    \item \textbf{Description}: Upload technical meeting, highlight code discussions, extract entities (e.g., project names), and generate technical report.
    \item \textbf{Preconditions}: Uploaded MP4 file, user has contributor role.
    \item \textbf{Postconditions}: Transcript indexed for search, report saved in Export \& Share.
    \item \textbf{UI Section}: Upload Meeting, Analytics Dashboard, Semantic Search.
    \item \textbf{Steps}:
        \begin{enumerate}
            \item User uploads MP4 file via "Record Meeting".
            \item System transcribes and identifies technical content.
            \item User views entity extraction in Analytics Dashboard.
            \item System indexes content for Semantic Search.
        \end{enumerate}
    \item \textbf{Exceptions}: File size exceeds 1GB, triggers chunked processing with user notification.
\end{itemize}

\section{Diagrams}
\subsection{Use Case Diagram}
\begin{figure}[H]
    \centering
    \includegraphics[width=0.8\textwidth]{use_case_diagram.png}
    \caption{Use Case Diagram showing actors (HR, Engineers, Admins) and use cases (Record Meeting, Analyze Sentiment, Manage Users).}
\end{figure}

\subsection{Sequence Diagram}
\begin{figure}[H]
    \centering
    \includegraphics[width=0.8\textwidth]{sequence_diagram_transcription.png}
    \caption{Sequence Diagram for Transcription Process (complete view).}
    \begin{subfigure}{0.3\textwidth}
        \includegraphics[width=\textwidth]{sequence_part1.png}
        \caption{Part 1: User Input}
    \end{subfigure}
    \begin{subfigure}{0.3\textwidth}
        \includegraphics[width=\textwidth]{sequence_part2.png}
        \caption{Part 2: API Call to AssemblyAI}
    \end{subfigure}
    \begin{subfigure}{0.3\textwidth}
        \includegraphics[width=\textwidth]{sequence_part3.png}
        \caption{Part 3: Output Display in UI}
    \end{subfigure}
\end{figure}

\subsection{Activity Diagram}
\begin{figure}[H]
    \centering
    \includegraphics[width=0.8\textwidth]{activity_diagram_record_meeting.png}
    \caption{Activity Diagram for Record Meeting use case, including decision points for pause/resume.}
\end{figure}

\subsection{Class Diagram}
\begin{figure}[H]
    \centering
    \includegraphics[width=0.8\textwidth]{class_diagram.png}
    \caption{Class Diagram showing Meeting (attributes: id, timestamp), Speaker (attributes: name, emotion), Transcript (attributes: text, entities) classes with relationships.}
\end{figure}

\subsection{Domain Model Diagram}
\begin{figure}[H]
    \centering
    \includegraphics[width=0.8\textwidth]{domain_model.png}
    \caption{Domain Model showing relationships between Meeting, User, Analytics, and Export entities.}
\end{figure}

\subsection{Data Flow Diagram (DFD)}
\begin{figure}[H]
    \centering
    \includegraphics[width=0.8\textwidth]{dfd_level0.png}
    \caption{Level 0 DFD showing system context with external entities (User, AssemblyAI).}
\end{figure}
\begin{figure}[H]
    \centering
    \includegraphics[width=0.8\textwidth]{dfd_level1_transcription.png}
    \caption{Level 1 DFD for Transcription Process with sub-processes.}
\end{figure}
\begin{figure}[H]
    \centering
    \includegraphics[width=0.8\textwidth]{dfd_level2_export.png}
    \caption{Level 2 DFD for Export Process with detailed data flows.}
\end{figure}

\subsection{Component Diagram}
\begin{figure}[H]
    \centering
    \includegraphics[width=0.8\textwidth]{component_diagram.png}
    \caption{Component Diagram showing Streamlit UI, FastAPI API, Celery Worker, and Redis Cache.}
\end{figure}

\subsection{Package Diagram}
\begin{figure}[H]
    \centering
    \includegraphics[width=0.8\textwidth]{package_diagram.png}
    \caption{Package Diagram organizing modules (UI, Processing, Analytics, Storage).}
\end{figure}

\subsection{Deployment Diagram}
\begin{figure}[H]
    \centering
    \includegraphics[width=0.8\textwidth]{deployment_diagram.png}
    \caption{Deployment Diagram showing hardware nodes (Server, Client) and software deployment (Docker containers).}
\end{figure}

\section{Implementation Challenges}
\subsection{NumPy/Pandas Compatibility}
\begin{itemize}
    \item \textbf{Problem}: Binary incompatibility between NumPy 2.2.6 and Pandas 2.2.2 caused runtime errors during analytics computation.
    \item \textbf{Solution}: Pinned versions to NumPy 1.24.3 and Pandas 2.1.3, implemented a compatibility script to detect version mismatches, and updated \texttt{requirements.txt}.
    \item \textbf{Impact}: Ensured stability for large dataset processing, reduced crash rate by 95%.
    \item \textbf{Lesson Learned}: Regular dependency audits are critical for long-term maintenance.
\end{itemize}

\subsection{Video Recording Integration}
\begin{itemize}
    \item \textbf{Problem}: Threading conflicts and memory leaks occurred during simultaneous audio/video capture, leading to application crashes.
    \item \textbf{Solution}: Implemented dual threading with OpenCV (video) and PyAudio (audio), added resource cleanup with \texttt{gc.collect()}, and limited buffer size to 1024 chunks.
    \item \textbf{Impact}: Achieved synchronized recording with 99% success rate, reduced memory usage by 30%.
    \item \textbf{Lesson Learned}: Preemptive testing with edge cases (e.g., low RAM) improves reliability.
\end{itemize}

\subsection{Sentiment Model Accuracy}
\begin{itemize}
    \item \textbf{Problem}: Initial sentiment analysis models (YAMNet/CREMA-D) achieved only 75% accuracy on diverse audio inputs due to background noise and speaker variability.
    \item \textbf{Solution}: Enhanced models with pre-processing (noise reduction, normalization), fine-tuned with 500 hours of labeled meeting data, and integrated confidence scoring (0-100%).
    \item \textbf{Impact}: Improved accuracy to 90% across test cases, with real-time adjustments based on audio quality.
    \item \textbf{Lesson Learned}: Continuous model training with domain-specific data is essential for accuracy in dynamic environments.
\end{itemize}

\section{Appendix}
\subsection{Sample Transcripts}
See IMPLEMENTATION_CHALLENGES.md for detailed examples, including speaker diarization and key point extraction.

\subsection{Configuration Settings}
\begin{lstlisting}
RECORDING_CONFIG = {
    'sample_rate': 16000,
    'channels': 1,
    'chunk_size': 1024,
    'max_duration': 3600,  # 1 hour
    'video_fps': 30,
    'video_resolution': (640, 480),
    'video_codec': 'XVID'
}
ASSEMBLYAI_API_KEY = os.getenv('ASSEMBLYAI_API_KEY')
SMTP_CONFIG = {
    'server': 'smtp.gmail.com',
    'port': 587,
    'user': os.getenv('EMAIL_USER'),
    'password': os.getenv('EMAIL_PASSWORD')
}
\end{lstlisting}

\end{document}